\chapter{The Screens}
As stated before, Little Sound Dj has several screens, laid out in a screen map of size \begin{math} 5 \times 3 \end{math}. You can
navigate between the screens by pressing \textsc{select+cursor}.
\section{Screen Map}
\begin{figure}[htbp]
\centering
\begin{tabular}{|c|c|c|c|c|l}

	\cline{0-4}
	\multicolumn{2}{|c|}{ } & & & &
	\\
	\multicolumn{2}{|c|}{Project} & Wave & Synth & &
	\\
	\multicolumn{2}{|c|}{ } & & & &
	\\
	\cline{0-3}
	& & & & & \\
	Song & Chain & Phrase & Instr. & Table  & \begin{math} \leftarrow \end{math} Main Row \\
	& & & & & \\
	\cline{0-4}
	\multicolumn{5}{|c|}{} \\
	\multicolumn{5}{|c|}{Groove} \\
	\multicolumn{5}{|c|}{} \\
	\cline{0-4}
\end{tabular}
%\caption{Screen Map.}
\end{figure}

The song, chain and phrase screens are used for sequencing and arranging. The wave, synth,
instrument and table screens are used for sound programming. \footnote{There are also three hidden
screens, not shown on the map: The file, word and help screens. We will get back to these later.}

The remaining screens, project and groove, have more general purposes. The bulk of your activities will however probably be in the so-called "main row," in the middle of the map, as that's where the composing is done.

\section{Starting and Stopping}

When pressing \textsc{start} in the song screen, Little Sound Dj will always try to play all four
channels. When pressing \textsc{start} in the other screens, Little Sound Dj will only try to play the
channel that's indicated in the three-letter field at the right edge of the screen (\textsc{pu1}, \textsc{pu2}, \textsc{wav} or \textsc{noi}).

If you want to start playing all four channels from some other screen than the song screen,
you can do that by pressing \textsc{select+start}.

\section{Song Screen}

\begin{figure}[hbtp]
\centering
\fbox{ \includegraphics{song} }
\caption{Song Screen}
%\label{fig:song}
\end{figure}

The song screen
%(figure~\ref{fig:song})
is the highest level of the sequencer. This is where you arrange your songs.

The screen contains four columns, one for each channel. The columns contain lists of chains, which will be played from top to down. Different chains are used for different channels.

To insert a chain, move the cursor to an empty step and press \textsc{a}. If you want to add a new
chain, press \textsc{a} twice. To edit a chain, move the cursor to the chain number and press
\textsc{select+right}. To remove a chain, you can either press \textsc{a} twice or
press \textsc{b+a}.

To start or stop playing all channels in the song screen, press \textsc{start}. To instantly re-start all channels in the song screen, press \textsc{select+start} (this has the same effect as pressing \textsc{start, start} quickly).

\includegraphics[width=1cm]{tip}TIP!
\begin{itemize}
\item \textit{You can pull up down-below chains by pressing \textsc{b+a} on an empty step.}
\item \textit{\textsc{b+up/down} does page up/down.}
\item \textit{You can add or remove song screen bookmarks by tapping \textsc{b} three times \textsc{(b, b, b)}. This will shade the area under the cursor.}
%	\marginpar{\includegraphics[width=1cm]{tip}TIP!}
\end{itemize}

The number of rows in the song screen is limited to 255 (\$00-\$FE).

\section{Chain Screen}
Chains are used for stringing phrases together, thus creating a unit built out of many
phrases. A chain can represent a longer rhythm block, a melody or a bass line.

The chain screen contains two columns. The first column contains the list of phrases that are
to be stringed together, while the second column transposes the phrase on the same row.

\begin{figure}[hbtp]
\centering
\fbox{ \includegraphics{chainexample} }
\caption{Chain Screen}
\label{fig:chainexample}
\end{figure}

Example:
The chain in figure~\ref{fig:chainexample} would play phrase 3, adding 5 semitones to each note, and then play each of the phrases 4, 5, 6, without transposing.

To add a phrase to the chain, move the cursor to an empty step and press \textsc{a}. If you want to
insert a new phrase, press \textsc{a} twice. To edit a phrase, move the cursor to the phrase number
and press \textsc{select+right}.

When editing a chain, you can go to the chain in a neighboring channel by pressing \textsc{b+left/right}. It is also possible to go to the next or previous chain in the song screen by
pressing \textsc{b+up/down}.

The different channels all share the same set of chains; that is, no chain is ever assigned to a
specific channel. The number of chains is limited to 128 (\$00-\$7F).

\section{Phrase Screen}

\begin{figure}[hbtp]
\centering
\fbox{ \includegraphics{phrase} }
%\caption{Phrase Screen}
%\label{fig:phrase}
\end{figure}

The phrase screen is the most fundamental part of the sequencer; this is where you enter the actual note data. The phrase screen has four columns: the note column, the instrument column, and the command and command value columns.

The different channels all share the same set of phrases; that is, any phrase may be played back on any channel. A phrase might however sound very different, depending on the channel it is played back on. Example: If you have programmed a phrase to play a melody using a pulse instrument, that phrase can be played back in either of the pulse channels with good results, but it usually doesn't make sense to play back the phrase in the wave or noise channels.

The note column may look different depending on which instrument type is used. Most instruments present the note followed by octave. Instruments that play back samples (\textsc{kit}, \textsc{speech}) do however present the sample names instead.

The instrument column is used for selecting instruments. In total, you can use 64 different instruments, editable in the instrument screen.

\includegraphics[width=1cm]{tip}TIP!
\begin{itemize}
\item \textit{It is possible to change the pitch without retrigging the instrument by leaving the instrument column empty.}
%	\marginpar{\includegraphics[width=1cm]{tip}TIP!}
\end{itemize}

The command columns can be used to add effects to your phrase. For example, the K command kills the sound on the channel.

The number of phrases is limited to 255 (\$00-\$FE). The number of the phrase that is being edited is displayed in the top left corner of the screen.

\includegraphics[width=1cm]{tip}TIP!
\begin{itemize}
\item \textit{All phrases are 16 steps long by default, but it is also possible to set a shorter length by using the H (hop) command.}
%	\marginpar{\includegraphics[width=1cm]{tip}TIP!}
\end{itemize}


\section{Instrument Screen}

\begin{figure}[hbtp]
\centering
\fbox{ \includegraphics{instr} }
\caption{Instrument Screen}
\label{fig:instr2}
\end{figure}

There are five types of instruments available:

\begin{description}
\item[\textsc{pulse}] This instrument type produces pulse waves, and is used in pulse channels 1 and 2.
\item[\textsc{wave}] This instrument type can play back waves synthesized using the synth screen. It is used in the wave channel.
\item[\textsc{kit}] This instrument type plays sampled kits, stored in \textsc{rom}. (The samples are stored in 4 bits, 11,468 kHz.) It is used in the wave channel.
\item[\textsc{noise}] This instrument type produces filtered noise, and is used in the noise channel.
\item[\textsc{speech}] This instrument is locked to instrument number \$40, and is used for programming speech. For learning how to generate speech, please read chapter \ref{speech-chapter}.
\end{description}

You can change the instrument type by going to the type row and pressing \textsc{a+cursor}.

Remember that instruments don't automatically play in the right channel. For example, if you want to use a kit instrument to play drum samples, you have to do the following:

\begin{enumerate}
\item Go to the song screen, move cursor to the wave column, and insert a new chain by tapping \textsc{a} twice.
\item	Edit the chain by pressing \textsc{select+right}.
\item	Insert a new phrase by tapping \textsc{a} twice.
\item	Edit the phrase by pressing \textsc{select+right}. Now, you have a new phrase that is mapped to the wave channel.
\item	Create a new instrument by moving the cursor to the instrument column and tapping \textsc{a} twice.
\item	Press \textsc{select+right} to edit the instrument.
\item	Change the instrument type to \textsc{kit}.
\item	Go back to the phrase screen to start using your new instrument.
\end{enumerate}

\includegraphics[width=1cm]{tip}TIP!
\begin{itemize}
	\item \textit{In the instrument screen, press \textsc{select+b} to copy instruments and \textsc{select+a} to paste.}
%\marginpar{\includegraphics[width=1cm]{tip}TIP!}
\end{itemize}

\subsection{General Instrument Parameters}

These parameters are used in most instrument types.

\begin{description}
	\item[\textsc{name}] Name the instrument by pressing \textsc{a}. This is useful for keeping track of your instruments. The instrument name will also be shown in the border when selecting instruments in the phrase screen.
	\item[\textsc{type}] Use this to specify the instrument type.
	\item[\textsc{length}] Change the sound length.
	\item[\textsc{pan}] Pan the sound to left/right/both/none speakers. (Use the headphone output to hear the difference!)
	\item[\textsc{vib. type}] Change the effect and speed of the vibrato (V), pitch bend (P) and slide (L) commands. The high frequency (HF) setting can create very interesting timbres. The other settings are more conventional, but just as useful. Pressing \textsc{a+up/down} changes the direction of the vibrato.
    \item[\textsc{transp.}] When \textsc{on}, the instrument will be affected by transposes.
	\item[\textsc{table}] If set to values other than \textsc{off}, Little Sound Dj will start running the specified table when a note is played. If you want to edit the table, press \textsc{select+right} to get to the table screen. If you want to use a new table, tap \textsc{a} twice.
	\item[\textsc{automate}] This option extends the table functionality. When automation is activated, Little Sound Dj advances through the tables by one step for each time the instrument is triggered.
\end{description}

\subsection{Pulse Instrument Parameters}

\begin{figure}[htpb]
	\begin{center}
\fbox{ \includegraphics{instr-pulse} }
	\end{center}
	\caption{Pulse Instrument Screen}
	\label{fig:instr-pulse}
\end{figure}

\begin{description}
	\item[\textsc{envelope}] The first digit sets initial amplitude (0-\$F); the second digit sets release (0, 8: none, 1-7: decrease amplitude, 9-\$F: increase amplitude).
	\item[\textsc{wave}] Choose the wave type to be used.
	\item[\textsc{sweep}] Modulate the frequency. This only works on pulse channel 1. See Sweep/Shape (\textsc{s}) command documentation for further information.
\end{description}

\label{detune}
The detune settings can be used to create interesting phase effects, when the same phrase is played on both pulse channels:

\begin{description}
	\item[\textsc{pu2 tune}] Detune pulse channel 2 in semitones.
	\item[\textsc{pu fine}] Detune pulse channel 1 downwards, channel 2 upwards.
\end{description}

\subsection{Wave Instrument Parameters}

The wave instrument can play back synth sounds generated by the soft synthesizer found in the \textsc{synth} screen.

\begin{figure}[hbtp]
	\begin{center}
		\fbox { \includegraphics{instr-wave} }
	\end{center}
	%\caption{Wave Instrument Screen}
	%\label{fig:instr-wave}
\end{figure}

\begin{description}
	\item[\textsc{volume}] Set amplitude (0=0\%, 1=25\%, 2=50\%, 3=100\%)
	\item[\textsc{synth}] Select the synth sound to play back. To edit the synth sound being used, press \textsc{select+up} to go to the \textsc{synth} screen. If you want to use a new synth, tap \textsc{a} twice.
	\item[\textsc{play}] How to play back the synth sound: Once, loop, pingpong loop or manual. By selecting manual, only the first wave in the synth sound will be played, allowing you to step through the sound manually using the \textsc{f} command.
	\item[\textsc{length}] Set the length of the synth sound.
	\item[\textsc{repeat}] Set the loop point of the synth sound.
	\item[\textsc{speed}] Set how fast the synth sound should be played back.
\end{description}

\subsection{Kit Instrument Parameters}

\begin{figure}[hbtp]
	\begin{center}
	\fbox {	\includegraphics{instr-kit} }
	\end{center}
	%\caption{Kit Instrument Screen}
	%\label{fig:instr-kit}
\end{figure}

\begin{description}
	\item[\textsc{kit}] Choose the kits to use. The first kit will be used in the left note column in the phrase screen; the second kit will be used in the right note column in the phrase screen.
	\item[\textsc{pitch}] Pitch shift.
	\item[\textsc{offset}] Set the start loop point. If \textsc{loop} is set to \textsc{off}, this value can be used for skipping the initial part of a sound.
	\item[\textsc{len}] Set the sound length. (\textsc{aut}=always play the sample to its end.)
	\item[\textsc{loop}] Loop the sample. (\textsc{off}=don't loop, \textsc{on}=loop sound and start playing from custom offset, \textsc{atk}=loop sample and start playing from the beginning.)
	\item[\textsc{speed}] Select full speed or half speed.
	\item[\textsc{dist}] Select the algorithm that should be used when two kits are mixed together. \textsc{clip} is the default type. \textsc{shape} and \textsc{shape2} sound similar to \textsc{clip}, but with more high frequencies and less bass. \textsc{wrap} can be used to add some interesting digital distortion. When pressing \textsc{a+(left, left)} while \textsc{clip} value is selected, the program will jump out of range and play back sound from raw memory when clipping.
    \item[\textsc{p.speed}] Speed of \textsc{P} command.
\end{description}

\includegraphics[width=1cm]{tip}TIP!
\begin{itemize}
	\item \textit{For those running LSDj on emulator or with backup gear, there is a Java application for replacing the original sample kits available at} \url{http://littlesounddj.com/lsd/latest/lsd-patcher/}.
\end{itemize}

\subsection{Noise Instrument Parameters}

\begin{figure}[htpb]
	\begin{center}
		\fbox { \includegraphics{instr-noise} }
	\end{center}
	\caption{Noise Instrument Screen}
	\label{fig:instr-noise}
\end{figure}

\begin{description}
	\item[\textsc{envelope}] First digit is initial amplitude (0-\$F); second digit is release (0, 8: none, 1-7: decrease amplitude, 9-\$F: increase amplitude).
	\item[\textsc{shape}] Alter the noise shape. The first digit alters the pitch, the second
period alters the randomness.
	\item[\textsc{s cmd}] When set to \textsc{free}, altering noise shape by the \textsc{S}
command (\ref{command-shape-noise}) can in some circumstances\footnotemark mute the sound. When set to \textsc{stable}, the
\textsc{S} command is limited so that the sound will never be muted by accident.
\footnotetext{The exact circumstances for when sound can get muted is when a shape that ends with
digit 8-F is changed so that it ends with digit 0-7. In that case, the odds that
the sound will get muted is 1 out of 256.}
\end{description}

\subsection{Speech Instrument Parameters}

For information about how to generate speech, please read chapter \ref{speech-chapter}.

The number of instruments is limited to 64 (hexadecimal: \$00-\$39).

\section{Table Screen}

Tables are essentially sequences of transposes, commands and amplitude changes, which can be executed at any speed (by default, one tick per step) and applied to any channel. If you want to, you can assign tables to instruments (by changing the \textsc{table} setting in the instrument screen), so that a table will be started every time you play the instrument. It is the key to creating truly complex instruments in Little Sound Dj.

Tables contain six columns, which are executed from top to bottom.
The first column is the envelope column, by which it is possible to create custom amplitude envelopes. Next is the transpose column, that can be used to transpose the note being played by a given number of semitones. The other columns are command columns, just like the one in the phrase screen.

By default, each step will be executed in one tick, but it is also possible to select a different groove using the G (groove) command.

\includegraphics[width=1cm]{tip}TIP!
\begin{itemize}
	\item \textit{The transpose column has special functionality when using \textsc{kit} or \textsc{noise} type instruments. For \textsc{kit}, the transpose column works as a pitch shifter. For \textsc{noise}, the transpose column has the same effect as issuing the \textsc{s} (shape) command.}
\end{itemize}

\subsection{Custom Envelope Example}

The first digit in the envelope column sets the amplitude; the second digit sets for how many ticks that amplitude should remain.

\begin{figure}[htpb]
	\begin{center}
		\fbox{		\includegraphics{table-amp} }
	\end{center}
	\caption{Table Envelope Example}
	\label{fig:table-amp}
\end{figure}

The table in figure~\ref{fig:table-amp} creates an amplitude envelope with short attack and medium sustain. It could be used for a bass instrument.

\subsection{Arpeggio Example}

\begin{figure}[htpb]
	\begin{center}
		\fbox{ \includegraphics{table-arp}}
	\end{center}
	\caption{Arpeggio Example}
	\label{fig:table-arp}
\end{figure}

A typical use for tables is to create arpeggios. This is a musical term for playing notes very fast, so that the listener will get the impression that a chord is played. The table in figure~\ref{fig:table-arp} would emulate striking a major chord.

Shorter arpeggios can just as well be created using the C (chord) command in phrases (see \ref{command-chord} for example). Tables however still have to be used for creating longer arpeggios.

To view different tables, press \textsc{b+cursor}.

\includegraphics[width=1cm]{tip}TIP!
\begin{itemize}
	\item \textit{To make an instrument attack sound more interesting, it can be useful to let the first row in a table be transposed a few steps up or down.}
	\item \textit{There is a shortcut between the phrase and table screens. Press \textsc{select+right} on an A command in the phrase screen to edit the table selected with the A command. To jump back, press \textsc{select+left}.}
\end{itemize}

The number of tables is limited to 32 (\$00-\$1F).

\section{Groove Screen}

Grooves define the speed with which your phrases and tables are played back. They can be used for giving your songs some extra swing. The different sound channels do not need to be synchronized to each other; this means that you can use a separate groove for each phrase and table.

\begin{figure}[htbp]
	\begin{center}
		\fbox{ \includegraphics{groove} }
	\end{center}
	\caption{Groove Screen}
	\label{fig:groove}
\end{figure}

For understanding the groove concept, you need to know that the sequencer's time handling is based on an abstract time period called \emph{tick}. The length of a tick varies with the song tempo, but is typically around 1/60th of a second. In the groove screen, you can specify for how many ticks each note step should be played.
The groove in figure~\ref{fig:groove} would make the sequencer spend approximately 6/60th of a second on every note step.

\begin{figure}[htbp]
	\begin{center}
		\fbox{ \includegraphics{groove-swing} }
	\end{center}
	\caption{Swing Example}
	\label{fig:groove-swing}
\end{figure}

You can also use the groove screen to create custom rhythms. The groove in figure~\ref{fig:groove-swing} would make the sequencer spend 8/60th of a second on even note steps, and 5/60th of a second on odd note steps. This would create a swing feeling. With thoughtful programming, grooves can also be used to create triplets and other complex rhythm structures.

Groove 0 is the default groove for all phrases. If you want to, you can easily switch to another groove by using the groove (\textsc{g}) command in the phrase screen.

You can select the groove you wish to edit by pressing \textsc{b+cursor}.

\includegraphics[width=1cm]{tip}TIP!
\begin{itemize}
	\item \textit{ Pressing \textsc{a+up/down} will change the swing percentage, while keeping the total number of ticks -- and thus, the resulting song speed -- constant. (Example: Original value is 6/6 = 50\%. Press \textsc{a+up}. Now the value changes to 7/5 = 58\%!) }
	\item \textit{ If you switch to the groove screen when the cursor is on a \textsc{g} command in the phrase or table screens, Little Sound Dj will display the groove that is selected with the groove command.}
\end{itemize}

\section{Synth Screen}

The synth screen features a soft synthesizer that generates sounds to be played back by the wave instruments. In total, there are 16 synth programs. You can choose the program to edit by pressing \textsc{b+cursor}.

\begin{figure}[h]
\includegraphics[width=1cm]{tip}TIP!
\begin{itemize}
	\item \textit{Each synth program uses \$10 waves. Synth program 0 uses waves \$00-\$0F, synth program 1 uses waves \$10-\$1F, and so on. It is possible to look at the resulting synth sounds in the wave screen (Section~\ref{wave-screen-section}).}
\end{itemize}
\end{figure}

\begin{figure}[htbp]
	\begin{center}
		\fbox{		\includegraphics{synth}}
	\end{center}
	\caption{Synth Screen}
	\label{fig:synth}
\end{figure}

\subsection{General Parameters}

\begin{description}
\item[\textsc{wave}] Square, saw tooth or triangle.
\item[\textsc{filter}] Low-pass, high-pass, band-pass or all-pass.
\item[\textsc{q}] Resonance control. Boost the signal around the cutoff frequency, to change how bright or dull the wave sounds.
\item[\textsc{dist}] Use clip or wrap distortion.
\item[\textsc{phase}] \label{phase}
Compress the waveform horizontally. It is applied after filtering with Q and cutoff. See figure \ref{fig:phasing} for examples.
\end{description}

\begin{figure}[hbtp]
	\centering
	\subfloat[Phase example. Original wave.]{
		\fbox{\includegraphics{phase-original}}
	}
	\qquad
	\subfloat[\textsc{normal} phasing. Compress horizontally, generate once.]{
	\fbox{\includegraphics{phase-normal}}
	}

	\subfloat[\textsc{resync} phasing. Compress horizontally, loop.]{
		\fbox{\includegraphics{phase-resync}}
	}
	\qquad
	\subfloat[\textsc{resyn2} phasing. Loop, but don't compress.]{
		\fbox{\includegraphics{phase-resyn2}}
	}
	\caption{Phase Examples}
	\label{fig:phasing}
\end{figure}

\subsection{Start and End Parameters}

Use these settings to specify values for the start and end of the sound. The program will then create a smooth fade between the start and end values.

\begin{description}
\item[\textsc{volume}] Wave volume.
\item[\textsc{cutoff}] Filter cutoff frequency.
\item[\textsc{phase}] 0 = no phase, \$1F = maximum phase. See figure \ref{fig:phasing} for examples.
\item[\textsc{vshift}] Shift the waveform vertically. See figure \ref{fig:vshift} for examples.
\end{description}

\begin{figure}[htpb]
	\centering
	\subfloat[Vshift example. Original wave.]{
		\fbox{\includegraphics{vshift-original}}
	}

	\subfloat[Vshifted wave. Vshift set to 40.]{
	\fbox{\includegraphics{vshift-40}}
	}

	\subfloat[Vshifted wave. Vshift set to 80.]{
		\fbox{\includegraphics{vshift-80}}
	}
	\caption{Vshift Examples}
	\label{fig:vshift}
\end{figure}

\section{Wave Screen}
\label{wave-screen-section}

%\begin{figure}[htpb]
%	\begin{center}
%		\fbox{		\includegraphics{wave}}
%	\end{center}
%	\caption{Wave Screen}
%	\label{fig:wave}
%\end{figure}
In the wave screen, you can view and edit the individual waveforms of the synth programs. There are 16 (\$10) synth programs, and each programs has \$10 waves. This means that synth sound 0 uses waves \$0-\$F, synth sound 1 uses waves \$10-\$1F, and so on.

To change a value, press \textsc{up/down}. To flip a value vertically, press \textsc{a+up/down}.
\textsc{B+cursor} navigates between different waves.

It is possible to modify multiple values at once, using the regular key presses:

\begin{description}
	\item[\textsc{select+b}] Start selection.
	\item[\textsc{select+b,b}] Select the entire wave.
	\item[\textsc{b}] Copy selection to clipboard.
	\item[\textsc{up/down}] Move selection up/down.
	\item[\textsc{a+up/down}] Flip selection vertically.
	\item[\textsc{select+a}] Paste from clipboard.
\end{description}

\section{Project Screen}

\begin{figure}[htpb]
	\begin{center}
		\fbox{		\includegraphics{project}}
	\end{center}
	\caption{Project Screen}
	\label{fig:project}
\end{figure}

The project screen (figure~\ref{fig:project}) contains settings that affect the entire program.

\begin{description}
	\item[\textsc{tempo}] Change the tempo. It is possible to set a new tempo either by pressing
\textsc{a+cursor}, or by tapping the \textsc{a} button in pace with the desired tempo. When being
slave in sync mode, it is possible to temporarily play a little faster or slower by pressing \textsc{a+left/right}.
	\item[\textsc{transpose}] Adjust the pitch of the pulse and wave instruments, by a given number of semitones.
	\item[\textsc{sync}] Activate link-up over the serial port. (Read more about this in chapter \ref{sync-chapter}!)
	\item[\textsc{clone}] Select deep or slim chain cloning. Deep chain cloning will also clone the phrases of a chain when cloning, whereas slim cloning will re-use the old phrases. Read chapter 3 for a full explanation of cloning.
	\item[\textsc{look}] Change the font and color set.
	\item[\textsc{key delay}] Set the delay time before key repeat is activated for the Game Boy buttons.
	\item[\textsc{key repeat}] Set the key repeat speed for the Game Boy buttons.
	\item[\textsc{prelisten}] Play notes and instruments while entering them.
	\item[\textsc{help}] Enter help screen. The help screen contains a quick reference for button presses and a command list.

	\item[\textsc{clean song data}] Clear all phrases and chains that are not used in the song. Also, if there are several phrases with the same content, they will be reduced to one. \label{clean-song-data}
	\item[\textsc{clean instr data}] Clear all instruments, tables, synths and waves that are not used in the song.
	\item[\textsc{load/save song}] Enter file manager. \footnote{The file manager is only available for cartridges that have 1 Mbit SRAM or more. In case your cartridge doesn't have 1 Mbit SRAM, this button will be replaced with a \textsc{reset memory} button.}
\end{description}

This screen also contains two clocks. The \textsc{work time} clock displays the time Little Sound Dj has been used since the last memory reset, in hours and minutes. When playing, the clock
is replaced by a \textsc{play time} clock, which shows for how long the song has been playing. The \textsc{total} clock displays the time Little Sound Dj has been used in total, in days, hours and minutes.

\subsection{Total Memory Reset}
\label{total-memory-reset}

By pressing \textsc{select+a} on the \textsc{load/save file} button, you can choose to reset all memory. This can be useful if your memory somehow gets scrambled, or your cartridge starts to behave strangely in other ways.

\section{File Screen}

\begin{figure}[htpb]
	\begin{center}
		\fbox{\includegraphics{file}}
	\end{center}
	\caption{File Screen}
	\label{fig:file}
\end{figure}

The file screen (figure~\ref{fig:file}) is entered by pressing the \textsc{load/save file} button in the project screen. The file screen is used for saving the song you are working on to the storage memory. It can also be used to load songs from the storage memory to the work memory. The file screen allows you to keep up to 32 songs on one cartridge.

\textsc{Note}: The file screen is only available for cartridges that have 1 Mbit SRAM or more.

\begin{description}
	\item[\textsc{file}] Shows the file name of the song you are working on. The exclamation mark (\textsc{!}) indicates when changes have been made to a song.
	\item[\textsc{load}] Load a song. Press \textsc{a}, select the file to load and press \textsc{a} again.
	\item[\textsc{save}] Save song. Press \textsc{a}, select the slot to save to and enter the file name.
	\item[\textsc{del}] Delete a song. Press \textsc{a}, select the file to delete and press \textsc{a} again.
	\item[\textsc{blocks used}] Shows how much of the storage memory that is used. One block equals 512 bytes. The digits on the bottom are hexadecimal, meaning there is a total of \$BF * 512 = 97,792 available bytes.
\end{description}

If you want to cancel an operation in this screen, simply press \textsc{b}.

\begin{figure}[hbtp]
\includegraphics[width=1cm]{tip}TIP!
\begin{itemize}
        \item \textit{There is a useful file manager application available at} \url{http://littlesounddj.com/lsd/latest/lsd-manager/}.
	\end{itemize}
\end{figure}

\subsection{Song List}

The song list presents song name, version number and file size. When saving, the song is compressed, so the resulting file size will vary with different songs. If you want to start a new project, load from the \textsc{(empty)} slot.

\includegraphics[width=1cm]{tip}TIP!
\begin{itemize}
    \item{While in the song list, it is possible to press \textsc{select+a} to load a song without switching to the song screen, and \textsc{start} to start/stop songs. In this way, you can load and play songs without jumping back and forth between screens. This can be handy if you are playing a live show with prepared tracks and want fewer things to think about.}
\end{itemize}

\section{Border Information}

A lot of useful data is displayed in the screen border (figure~\ref{fig:border}).

\begin{figure}[htpb]
	\begin{center}
	\includegraphics[width=12cm]{border}
	\end{center}
	\caption{Border Information}
	\label{fig:border}
\end{figure}

\begin{enumerate}
\item Screen name.
\item Phrase/chain/instrument/table/frame/groove number.
\item Active channel.
\item Chain position being edited.
\item Current tempo, in beats per minute (\textsc{bpm}).
\item Notes being played.
\item Sync information.
\item Mute. (The characters will be lit when pressing \textsc{b+select} or \textsc{b+start}.)
\item Screen map.
\end{enumerate}


